\documentclass[12pt,a4paper]{article}
\usepackage{fontspec}
\usepackage{polyglossia}
\setdefaultlanguage{ukrainian}
\setmainfont{Times New Roman}
\setsansfont{Arial}
\usepackage{graphicx}
\usepackage{amsmath}
\usepackage{geometry}
\usepackage{booktabs}
\usepackage{float}
\usepackage{caption}
\captionsetup{font=small,labelfont=bf}
\usepackage{microtype}

\geometry{margin=2.5cm}

\title{Аналіз жестів руки: k-NN та k-means}
\author{Шпагін Павло}
\date{\today}

\begin{document}

\maketitle

У роботі розглядається класифікація та кластеризація жестів руки за ознаками ширини й висоти обмежувальної рамки. На синтетично згенерованих даних застосовано алгоритми k-NN та k-means без використання готових бібліотек. Досліджується вплив третього класу (знак перемоги) на якість розпізнавання.

\section{Постановка задачі}

\textbf{Об'єкт дослідження:} гіпотетичне зображення руки.

\textbf{Вектор ознак:} $(w, h)$ -- ширина та висота долоні (см).

\textbf{Класи:} Кулак, Долоня (розкрита).

\section{Генерація даних}

Дані згенеровано з нормального розподілу для двох базових жестів:
\textbf{Кулак} ($w \sim N(8, 0.8)$, $h \sim N(4, 0.9)$; 30 зразків) та
\textbf{Долоня} ($w \sim N(16, 1.5)$, $h \sim N(16, 1.8)$; 30 зразків).

\begin{figure}[H]
    \centering
    \includegraphics[width=\textwidth]{box_examples.png}
    \caption{Приклади обмежувальних рамок для жестів руки}
\vspace{-0.4cm}
\end{figure}

\begin{figure}[H]
    \centering
    \includegraphics[width=0.7\textwidth]{data_visualization.png}
    \caption{Розподіл даних у просторі ознак}
\vspace{-0.4cm}
\end{figure}

\section{k-NN класифікація}

Реалізовано алгоритм k-NN без використання бібліотек. Відстань -- евклідова:
\begin{equation}
    d(p_1, p_2) = \sqrt{(w_1-w_2)^2 + (h_1-h_2)^2}
\end{equation}

\subsection*{Експериментальна перевірка}
Додатково проведено перевірку точності на \textbf{окремо згенерованих тестових даних} (навчальна та тестова вибірки
генерувалися незалежно). Порівнювалися два сценарії: (i) класифікація двох класів (Кулак/Долоня) та
(ii) класифікація трьох класів після додавання ``Знаку перемоги'' до тренувальної і тестової вибірок.

\begin{table}[H]
\centering
\begin{tabular}{lcc}
\toprule
Сценарій & k & Точність \\
\midrule
2 класи (Кулак/Долоня) & 1 & 100.0\% \\
2 класи (Кулак/Долоня) & 3 & 100.0\% \\
2 класи (Кулак/Долоня) & 5 & 100.0\% \\
2 класи (Кулак/Долоня) & 7 & 100.0\% \\
\midrule
3 класи (+ Знак перемоги) & 1 & 98.9\% \\
3 класи (+ Знак перемоги) & 3 & 98.9\% \\
3 класи (+ Знак перемоги) & 5 & 98.9\% \\
3 класи (+ Знак перемоги) & 7 & 98.9\% \\
\bottomrule
\end{tabular}
\caption{Точність k-NN на незалежних тестових вибірках для 2 та 3 класів}
\end{table}

\begin{figure}[H]
    \centering
    \includegraphics[width=0.7\textwidth]{knn_boundary.png}
    \caption{Межа рішень k-NN (k=3)}
\vspace{-0.4cm}
\end{figure}

\section{Проблема класу ``Знак перемоги''}

Якщо збільшити кількість класів у задачі k-NN з 2 до 3 та додати клас \textbf{``Знак перемоги''}
($w \sim N(10, 1.5)$, $h \sim N(12, 2.0)$), то класифікація в просторі ознак $(w, h)$ ускладнюється:
у деяких випадках області класів можуть перекриватися, а ``найближчі сусіди'' для нового жесту
виявляються змішаними.

\begin{figure}[H]
    \centering
    \includegraphics[width=0.7\textwidth]{victory_problem.png}
    \caption{Перекриття класів при додаванні знаку перемоги}
\vspace{-0.4cm}
\end{figure}

\textbf{Практичне вирішення для k-NN} полягає в тому, щоб додати приклади жесту \textbf{``Знак перемоги''}
до \textbf{тренувальної вибірки} як окремий клас. Алгоритм k-NN є багатокласовим за своєю природою
(через голосування сусідів), тому він може працювати і з 3+ класами за умови, що тренувальні дані
містять приклади кожного класу. З експериментальної перевірки видно, що після додавання третього класу
якість залежить від вибору $k$ і від перекриття розподілів у просторі ознак.

Водночас, використання лише 2D-ознак $(w, h)$ є обмеженням: вони описують розмір, але не геометрію жесту.
Для підвищення якості розпізнавання доцільно розширювати вектор ознак (наприклад, кількість пальців,
кути між пальцями, співвідношення сторін, морфологічні характеристики контурів).

\section{k-means кластеризація}

\begin{figure}[H]
    \centering
    \includegraphics[width=\textwidth]{kmeans_results.png}
    \caption{Результати k-means для k=2,3,5}
\vspace{-0.4cm}
\end{figure}

\begin{table}[H]
\centering
\begin{tabular}{cc}
\toprule
k & WCSS \\
\midrule
2 & 196.0 \\
3 & 131.4 \\
5 & 96.6 \\
\bottomrule
\end{tabular}
\caption{Within-Cluster Sum of Squares}
\end{table}

\section{Висновки}

Отримані результати показують, що k-NN ефективно розділяє два базові класи у 2D просторі ознак.
Після додавання третього класу точність залишається високою, але стає більш чутливою до вибору параметра $k$
та до перекриття класів. Метод k-means підтверджує природну двокластерну структуру базових даних (k=2).
Загалом, для класифікації більшої кількості жестів доцільно використовувати розширений набір ознак,
який описує конфігурацію пальців, а не лише габаритні розміри.

\end{document}
