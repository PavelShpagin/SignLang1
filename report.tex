\documentclass[12pt,a4paper]{article}
\usepackage{fontspec}
\usepackage{polyglossia}
\setdefaultlanguage{ukrainian}
\setmainfont{Times New Roman}
\setsansfont{Arial}
\usepackage{graphicx}
\usepackage{amsmath}
\usepackage{geometry}
\usepackage{booktabs}
\usepackage{float}
\usepackage{caption}
\captionsetup{font=small,labelfont=bf}

\geometry{margin=2.5cm}

\title{Аналіз жестів руки: k-NN та k-means}
\author{Лабораторна робота}
\date{\today}

\begin{document}

\maketitle

\section{Постановка задачі}

\textbf{Об'єкт дослідження:} гіпотетичне зображення руки.

\textbf{Вектор ознак:} $(w, h)$ -- ширина та висота долоні (см).

\textbf{Класи:} Кулак, Долоня (розкрита).

\section{Генерація даних}

Дані згенеровано з нормального розподілу:
\begin{itemize}
    \item \textbf{Кулак:} $w \sim N(8, 0.8)$, $h \sim N(4, 0.9)$ -- 30 зразків
    \item \textbf{Долоня:} $w \sim N(16, 1.5)$, $h \sim N(16, 1.8)$ -- 30 зразків
\end{itemize}

\begin{figure}[H]
    \centering
    \includegraphics[width=\textwidth]{box_examples.png}
    \caption{Приклади обмежувальних рамок для жестів руки}
\vspace{-0.4cm}
\end{figure}

\begin{figure}[H]
    \centering
    \includegraphics[width=0.7\textwidth]{data_visualization.png}
    \caption{Розподіл даних у просторі ознак}
\vspace{-0.4cm}
\end{figure}

\section{k-NN класифікація}

Реалізовано алгоритм k-NN без використання бібліотек. Відстань -- евклідова:
\begin{equation}
    d(p_1, p_2) = \sqrt{(w_1-w_2)^2 + (h_1-h_2)^2}
\end{equation}

\begin{table}[H]
\centering
\begin{tabular}{cc}
\toprule
k & Точність \\
\midrule
1 & 100\% \\
3 & 100\% \\
5 & 100\% \\
7 & 100\% \\
\bottomrule
\end{tabular}
\caption{Результати k-NN для різних k}
\end{table}

\begin{figure}[H]
    \centering
    \includegraphics[width=0.7\textwidth]{knn_boundary.png}
    \caption{Межа рішень k-NN (k=3)}
\vspace{-0.4cm}
\end{figure}

\section{Проблема класу "Знак перемоги"}

Якщо для k-NN збільшити кількість класів від 2 до 3 і додати клас \textbf{``Знак перемоги''}
($w \sim N(10, 1.5)$, $h \sim N(12, 2.0)$), то виникають проблеми через перекриття у просторі ознак:

\begin{figure}[H]
    \centering
    \includegraphics[width=0.7\textwidth]{victory_problem.png}
    \caption{Перекриття класів при додаванні знаку перемоги}
\vspace{-0.4cm}
\end{figure}

\textbf{Недоліки:}
\begin{enumerate}
    \item \textbf{Перекриття кластерів:} Знак перемоги має проміжні розміри
    \item \textbf{Недостатність ознак:} Ширина/висота не описують конфігурацію пальців
    \item \textbf{Неоднозначність:} Різні жести можуть мати однакові розміри
\end{enumerate}

\textbf{Рішення:} Додати ознаки -- кількість пальців, кути між пальцями, співвідношення сторін.

\section{k-means кластеризація}

\begin{figure}[H]
    \centering
    \includegraphics[width=\textwidth]{kmeans_results.png}
    \caption{Результати k-means для k=2,3,5}
\vspace{-0.4cm}
\end{figure}

\begin{table}[H]
\centering
\begin{tabular}{cc}
\toprule
k & WCSS \\
\midrule
2 & 196.0 \\
3 & 131.4 \\
5 & 96.6 \\
\bottomrule
\end{tabular}
\caption{Within-Cluster Sum of Squares}
\end{table}

\section{Висновки}

\begin{enumerate}
    \item \textbf{k-NN:} Ефективний для двох класів (Кулак/Долоня) з точністю 100\%
    \item \textbf{k-means:} k=2 найкраще відповідає природній структурі даних
    \item \textbf{Обмеження:} 2D ознаки (ширина, висота) недостатні для розрізнення складніших жестів
    \item \textbf{Рекомендація:} Для класифікації >2 жестів потрібні додаткові ознаки
\end{enumerate}

\end{document}
